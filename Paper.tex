\documentclass{article}


\usepackage{arxiv}

\usepackage[utf8]{inputenc} % allow utf-8 input
\usepackage[T1]{fontenc}    % use 8-bit T1 fonts
\usepackage{hyperref}       % hyperlinks
\usepackage{url}            % simple URL typesetting
\usepackage{booktabs}       % professional-quality tables
\usepackage{amsfonts}       % blackboard math symbols
\usepackage{nicefrac}       % compact symbols for 1/2, etc.
\usepackage{microtype}      % microtypography
\usepackage{lipsum}
\usepackage[section]{algorithm}
\usepackage{fixltx2e}
\usepackage{algpseudocode}
\usepackage{hyperref}
\usepackage{graphicx}
\usepackage{amsthm}
\usepackage{amsmath}
\usepackage{amssymb}
\usepackage{tikz-cd}
\usepackage{etoolbox}
%\usepackage{tabularx}
\usepackage{url}
\usepackage{amsmath}

% this is for environments \subfigure and \subtable
\usepackage{subcaption}

% These packages are FORBIDDEN
%%%%%%%%%%%%%%%%%%%%%%%%%%%%%%%%%%%%%%%%%%%%%%%%%%%%%%%%
%\usepackage{lmodern} % messes up \textsc
%\usepackage{cite} % messes up NIPS
%\usepackage{fullpage} % messes up NIPS
%\usepackage{natbib} % messes up NIPS
%%%%%%%%%%%%%%%%%%%%%%%%%%%%%%%%%%%%%%%%%%%%%%%%%%%%%%%%

\usepackage{array}          % replacement for eqnarray.  Must be BEFORE \usepackage{arydshln}
\usepackage{units}          % for \nicefrac{\alpha}{\beta}


\usepackage{amsthm}     % for theorems
\newtheorem{definition}{Definition}

% text looks a little better
\usepackage{microtype}

\usepackage{wasysym}

\usepackage{textcomp, marvosym} % pretty symbols
\usepackage{booktabs}   % for much better looking tables

% for indicator functions
\usepackage{dsfont}

% For automatic capitalizaton of section names, etc.
\usepackage{titlesec,titlecaps}
\usepackage{float} % for \subfloat

%%%%%%%%%%%%%%%%%%%%%%%%%%%%%%%%%%%%%%%%%%
%% More customizable Lists
%%%%%%%%%%%%%%%%%%%%%%%%%%%%%%%%%%%%%%%%%%
% Better symbols custom enumerative lists, define any symbol you'd like
% \usepackage{enumitem}


%%%%%%%%%%%%%%%%%%%%%%%%%%%%%%%%%%%%%%%%%%
%% Custom Symbols 
%%%%%%%%%%%%%%%%%%%%%%%%%%%%%%%%%%%%%%%%%%
% \xspace at the end of custom macros never screws up spacing.
\usepackage{xspace}
% for better manipulation of tables
\usepackage{makecell}
\renewcommand\theadfont{\bfseries}

\title{Bio-cryptocurencies
}


\author{
  Adan M. Pablo\thanks{Use footnote for providing further
    information about author (webpage, alternative
    address)---\emph{not} for acknowledging funding agencies.} \\
 \\\texttt{pabloadd2@gmail.com} \\
  %% examples of more authors
   \And
\\
\\
%% \AND
  %% Coauthor \\
  %% Affiliation \\https://es.overleaf.com/project/5daf4e704d26da000143e74c
  %% Address \\
  %% \texttt{email} \\
  %% \And
  %% Coauthor \\
  %% Affiliation \\
  %% Address \\
  %% \texttt{email} \\
  %% \And
  %% Coauthor \\
  %% Affiliation \\
  %% Address \\
  %% \texttt{email} \\
}

\begin{document}
\maketitle


\begin{abstract}
In the last decade, the research in

\end{abstract}


% keywords can be removed
\keywords{Blockchain \and DLT \and Cryptocurrency \and Consensus  \and Dsitrubuted systems \and DNA \and DNA cryptography\and Bio-inspired computation} 


\section{Introduction}
Brianstorm : UN PUNTO UNA REFENCIA
The "evolutionª" in cryptosystems 
.the DNA storage to development in a better cryptography().
when this two way of science work together is possible imagine a bio cryptocurrency.the genetic engineering needs to work in creating a lot of tools in the cell for develempoment a cryptosystem inside. A p2p cell comunication, hash functions, cryptography, and all the important systems that use the cryptocurrency\cite{article2}\cite{8014672}\cite{10.1093/rfs/hhy095}. so far this tools are in Moore´s law\cite{Schaller:1997:MLP:254613.254618} in a low develomnpent.When bizantine problems was partial solve by lamport at 1980´s \cite{Lamport:1982:BGP:357172.357176} the distributed systems was a open develompent of scalability and a good theorem math.The invention of cryptoeconomics by Satoshi Nakamoto\cite{nakamoto2012bitcoin} and the solution of double spend problem was the result of exploring the design phylosophy of protocols of distributed systems\cite{Back02hashcash-}\cite{Dang:2016:PMS:2935634.2935638}.The architure on bioinfomatics are so separted of the current algotithm computational on DLT technologies and blockchain uses cases.The design philoshopy and template for a consensus are in the trilemma (Descentralization, Security and Scalaebility) of blockchains\cite{RePEc:nbr:nberwo:25407}.Consensus in differentes cryptocurrecnies are Bizantine Fault Tolerance \cite{Castro:1999:PBF:296806.296824},Bizanttine Agremment\cite{DBLP:journals/corr/Micali16} and  survey , with an architure in Blockchains(blockheaders chain), DAG\cite{10.1007/978-3-030-13651-2_11} with funcionality in privacy\cite{inproceedingszcash} .The bootleneck escenario\cite{inproceedings}\cite{inproceedings2} are  in "chasm time" of the evolotuion of technologies\cite{Polhill2019}.The new version of cryptocurrency focus to the TPS thas is signal of economic growth in the ecosystem and a mesuare of --(). Layer 1 and Layer 2 are in continous developemnt in topic like intereperability, cross-chain conection and sidechains velocity. In layer 1 VDF and BLS signatues or another signature scheme. In layer 2 the zero knwolodge proof\cite{DBLP:conf/crypto/Ben-SassonBHR19} is a tool for privacy\cite{Kappos:2018:EAA:3277203.3277238} and scalaebility\cite{cryptoeprint:2019:550}. the first public design of cryprtocurrency in a living cell is Coinami\cite{Ileri2016CoinamiAC} that use a DNA  sequence aligment as Proof of work other experiments consist in put the private key in the cell (cold storage) in a simple encryption architure.so far those experiments that involves a living cell and the cryptocurrency ecosystem that are in date on the literature review.The first implemantion of data on a cell was in , since this novel invention been apper different tools and development how the dna store data inside like Crips-cas enconding\cite{Shipman2016MolecularRB} and DNA cryptography\cite{8212786}.The edition of genome without double stranded DNA breaks could develepmnet betther techniques in the artichure of code in the living cell (producing a higher editing selectivity)\cite{article}.In bio-Inspired \cite{Ser2019BioinspiredCW}... other experiments are the rasperry pi in a bacteria \cite{inproceedings423}and  "DNA-BOTS" \cite{Storch832139} are part of the internet of bio-nano things\cite{Akyildiz2015TheIO}.
\section{Theroem and math in cryptosystems}
The represantion of blocks that have been propouse in the chain
\begin{equation}
 \mathnorma{B^t= (B_1, \dotso }B_n)
 \end{equation}
We first assume that at any time t⁠, all miners observe the tree of solved blocks t={Bt,Et,It}⁠, where Bt=(B0,…Bn) is the set of all blocks that have been solved by time t⁠, Et={(B0,B1),…(Bk,Bk′),…}⁠, with 0≤k<k′≤n⁠, is the set of edges chaining these blocks, and It=(m(B1),…m(Bn)) is the set of identities of miners who solved block
A DLT is fundamental a monadic implemantion of concurrent mutations of a global state or a bizantine fault-tolerant state-machine replication in other cases.
let  be totally orderd, countaqble, set of posibloe states.
let represent a special, invalid , state.
let b ve the set of data. the set of valid data is  
The tuple of blockchain is:


\begin{equation}
( \mathbf{S} , \leq, \oslash, \mathbf{B} \subset \mathbf{S}^S∪\oslash)


\end{equation}

For Consensus safety the principal template in correct by contruction have 3 Lemmas that proof the theoreom that in the architure always choose and agree on the next block.
\begin{defn}[$\sigma_1 \sim \sigma_2$]

\begin{equation*}
\begin{tikzcd}
{}
  &
{}  
  &
{}  
  &
{}
  &
\sigma_3
  &
{}
  \\
{}
  &
\sigma_1 \sim \sigma_2
  \arrow[r,Leftrightarrow, "def"]
  &
\exists \sigma_3:(\sigma_1 \to \sigma_3) \land (\sigma_2 \to \sigma_3)
  &
{}  
  &
{}
  &
{}
  \\
{}  
  &
{}  
  &
{}
  &
\sigma_1
  \arrow[uur,""]
  &
{}
  &
\sigma_2
  \arrow[uul,""]
\end{tikzcd}
\end{equation*}
\end{defn}

\begin{thm}[Consensus safety]
\begin{description}
$$
\sigma_1 \sim \sigma_2 \implies \neg(S(p,\sigma_1) \land S(\neg{p},\sigma_2))
$$
\end{description}
\end{thm}
\begin{equation*}
\begin{tikzcd}
{}
  &
\sigma_3
  &
{}
  &
{}
  &
S(p,\sigma_3)
  &
{}
  &
{}
  \\ 
{}
  &
{}
  &
{}
  \arrow[r,Rightarrow,""]
  &
{}
  &
{}
  &
{}
  &
{}
  \\
{}
  &
{}
  &
{}
  \arrow[r,Rightarrow,""]
  &
{}
  &
{}
  &
{}
  &
{}
  \\
\sigma_1
  \arrow[uuur,""]
  &
{}
  &
\sigma_2
  \arrow[uuul,""]
{}
  &
S(p,\sigma_1)
  \arrow[uuur,Rightarrow,"\text{Lemma 1}"]
  \arrow[rr,Rightarrow,"\text{Lemma 1 $\circ$ Lemma 3}"]
  &
{}
  &
\neg{S(\neg{p},\sigma_2)}
  \arrow[uuul,Leftarrow,"\text{Lemma 3}"] 
\end{tikzcd}
\end{equation*}

The safety proof (illustrated above) uses Lemma 1 (Future safety) between $\sigma_1$ and $\sigma_3$, and Lemma 3 (Backwards consistency) for $\sigma_3$ and $\sigma_2$, to conclude that $S(p,\sigma_1) \implies \neg{S(\neg{p},\sigma_2)}$. As it turns out, this property is equivalent to $\neg{(S(p,\sigma_1)\land S(\neg{p},\sigma_2))}$.
%\clearpage
\begin{table}[bth]
\centering

   \begin{tabular}{l l}
	\toprule
	\textbf{Notation} & \textbf{Description} \\
	\midrule
	$s$ & the hash of any justified checkpoint (the ``source'') \\
	$t$ & any checkpoint hash that is a descendent of  $s$ (the ``target'') \\
	$\h(s)$ & the height of checkpoint $s$ in the checkpoint tree \\
	$\h(t)$ & the height of checkpoint $t$ in the checkpoint tree \\
	\signature & signature of $\left\langle s, t, \h(s), \h(t) \right\rangle$ from the validator $\upnu$'s private key \\
	\bottomrule
	\end{tabular}


\vspace{0.15in}
\caption{The schematic of a single \msgVOTE message denoted $\left\langle \upnu, s, t, \h(s), \h(t) \right\rangle$.}
\label{tbl:messages}
\end{table}
We define the following terms:
\begin{itemize}
\item A \emph{supermajority link} is an ordered pair of checkpoints $(a, b)$, also written $a \rightarrow b$, such that at least $\frac{2}{3}$ of validators (by deposit) have published votes with source $a$ and target $b$.  Supermajority links \emph{can skip checkpoints}, i.e., it's perfectly okay for $\h(b) > \h(a) + 1$.  \figref{fig:2c} shows three distinct supermajority links in red: $r \to b_1$, $b_1 \to b_2$, and $b_2 \to b_3$.

\item Two checkpoints $a$ and $b$ are called \emph{conflicting} if and only if they are nodes in distinct branches, i.e., neither is an ancestor or descendant of the other.

\item A checkpoint $c$ is called \emph{justified} if (1) it is the root, or (2) there exists a supermajority link $c^\prime \to c$ where $c^\prime$ is justified.  \figref{fig:2c} shows a chain of four justified blocks.

\item A checkpoint $c$ is called \emph{finalized} if it is justified and there is a supermajority link $c \to c^\prime$ where $c^\prime$ is a \emph{direct child} of $c$.  Equivalently, checkpoint $c$ is finalized if and only if: checkpoint $c$ is justified, there exists a supermajority link $c \to c^\prime$, checkpoints $c$ and $c^\prime$ are not conflicting, and $\h(c^\prime) = \h(c) + 1$.
\end{itemize}

\begin{figure}
\begin{mdframed}
\textsc{An individual validator $\upnu$ must not publish two distinct votes,}
\begin{equation*}
\left\langle \upnu, s_1, t_1, \h(s_1), \h(t_1)\right\rangle \hspace{0.5in} \textsc{and} \hspace{0.5in} \left\langle \upnu, s_2, t_2, \h(s_2), \h(t_2)\right\rangle \; ,
\end{equation*}
\textsc{such that either:}

\begin{enumerate}
   \item[\textbf{I.}] $\h(t_1) = \h(t_2)$.

   Equivalently, a validator must not publish two distinct votes for the same target height.
\end{enumerate}
\vspace{-0.15in}
\textsc{or}

\begin{enumerate}
   \item[\textbf{II.}] $\h(s_1) < \h(s_2) < \h(t_2) < \h(t_1)$.

   Equivalently, a validator must not vote within the span of its other votes.
\end{enumerate}
\end{mdframed}
\caption{The two Casper Commandments.  Any validator who violates either of these commandments gets their deposit slashed.}
\label{fig:commandments}
\end{figure}
\cite{caspervlad}

 \begin{algorithm}[tbh]
 \caption{Updatin the State}
We now have the language required to define the estimator for the blockchain consensus, the Greedy Heaviest-Observed Sub-Tree rule!
\begin{defn}[The Greedy Heaviest-Observed Sub-Tree (GHOST) fork-choice rule, $\mathcal{E}$]
\end{defn}

\begin{algorithm}[H]
 \KwData{A set of blocks, $M$}
 \KwResult{The block at the head of the fork choice}
 $b$ = Genesis Block

 \While{$b$ has children ($C(b,M)$ is nonempty)}{
  scores = dict()

  \For{each child of block $b$, $b' \in C(b,M)$}{
  scores[$b'$] = $\text{Score}(b', M)$
  }
  \eIf{scores has a unique maximum}{
    $b$ = argmax(scores)
  }{
      $b$ = the max score block with the lowest hash
  }
}
\Return{b}

\caption{The Greedy Heaviest-Observed Sub-tree Fork-choice rule, $\mathcal{E}$}
\end{algorithm}
 \begin{algorithmic}
 \State  \bf{Initialize:} \normalfont 
    \begin{itemize}
        \item \bf{A preferences:} \normalfont 
            \begin{itemize}
                \item 1 for A
                \item 2 for A
                \item 3 for A
            \end{itemize}
        \item \bf{B preferences::} \normalfont
            \begin{itemize}
                \item 1 for B
                \item 2 for B
                \item 3 for B
            \end{itemize}
        \item \bf{Nature sets the laws:} \normalfont
            \begin{itemize}
                \item The nature chooses a set of incentives $\mathcal{R}$.
            \end{itemize}
    \end{itemize}
 \State  
 \While {Set of Incentives $\mathcal{R}$ Exists}
 \State  
 \State \bf{Defend:} \normalfont 
    \begin{itemize}
        \item I
        \item II
        \item III
    \end{itemize}

 \State  
 \State \bf{Attack:} \normalfont 
    \begin{itemize}
        \item I
        \item II
        \item III
    \end{itemize}
 \State  
 \State \bf{Nature:} \normalfont 
    \begin{itemize}
        \item The Nature updates $\mathcal{R}$.
    \end{itemize}
  \State  
\EndWhile 
 \end{algorithmic}
\end{algorithm}

\begin{algorithm}
  \caption{Algorithm for finding server indices using OFG}

  \begin{algorithmic}
    \Statex \Comment { \%comment: servers[] contains the index of servers whose         data rate are sorted in descending order\%}
    \State servers[]= index(of all servers) 
    \State serverIndex[]=servers[0..K]
    \State linearlyIndependentServerIndex[]=0
    \State $[Z] \leftarrow 0$
    \For  {$i=0$ to $serverIndex.length$} 
    \Statex\Comment{ \%comment: find the equation corresponding the serverIndex        from the mapping at the File Server\%} 
    \State        $eqn= equation(serverIndex[i])$ 
    \Statex\Comment{ \%comment: try insert equation into Z using OFG\%} 

    \EndFor end for 
    \While{ ( linearlyIndependentServerIndex.length!=K ) } 
    \Statex\Comment{\%comment: remove all the server index which were not inserted in Z\%}  
    \State temp[]=serverIndex[]-linearlyIndependentServerIndex 
    \If{  (linearlyIndependentServerIndex.length=K) }
    \State break
    \EndIf  
    \EndWhile  
  \end{algorithmic}
\end{algorithm}
  
\lipsum[5]
\begin{equation}
\xi _{ij}(t)=P(x_{t}=i,x_{t+1}=j|y,v,w;\theta)= {\frac {\alpha _{i}(t)a^{w_t}_{ij}\beta _{j}(t+1)b^{v_{t+1}}_{j}(y_{t+1})}{\sum _{i=1}^{N} \sum _{j=1}^{N} \alpha _{i}(t)a^{w_t}_{ij}\beta _{j}(t+1)b^{v_{t+1}}_{j}(y_{t+1})}}
\end{equation}

\subsubsection{Headings: third level}
\lipsum[6]

\paragraph{Paragraph}
\lipsum[7]

\section{Examples of citations, figures, references and tables.}
\label{sec:others}
\lipsum[8] \cite{kour2014real,kour2014fast} and see \cite{hadash2018estimate}.

The documentation for \verb+natbib+ may be found at
\begin{center}
  \url{http://mirrors.ctan.org/macros/latex/contrib/natbib/natnotes.pdf}
\end{center}
Of note is the command \verb+\citet+, which produces citations
appropriate for use in inline text.  For example,
\begin{verbatim}
   \citet{hasselmo} investigated\dots
\end{verbatim}
produces
\begin{quote}
  Hasselmo, et al.\ (1995) investigated\dots
\end{quote}

\begin{center}
  \url{https://www.ctan.org/pkg/booktabs}
\end{center}


\subsection{In cell:}
\lipsum[10] 
//algorithm encryption in DNA
1. Begin:2. Read the plain text;3. Input a random number N;4. While (N! = 0) doa. Shuffle the value ofTable 1according to the value of n.b. Decrement N;5. End while6. Convert plaintext into 7-bit sequence usingTable 1.7. Input the 14 bit keya. Extract 7 bit key depending upon the first bit8. Perform XOR between the key and 7-bit sequence9. Divide the resulting bit sequence into blocks of 7 bits10. Now find out the position i of each block fromTable 111. Substitute each block with the block located at (127-i)thposition inTable 112. Divide the resulting bit sequence into two halves13. Left shift the first half and right shift the second half14. Interchange the two halves15. Divide the resulting bit sequence into blocks of 6 bits16. if length of the resulting bit sequence is not divisible by6, pad zero’s at the end of last block17. Generate ciphertext usingTable 2.18. End
\begin{algorithm}
\caption{Encryption Process}\label{euclid}
\begin{algorithmic}[1]

\State Begin
\State Read the plain text
\State Input a random number N;
\State While (N! = 0) do
\State
\State
\State
\State
\State
\State
\State
\State
\State
\State
\State
\State



\EndProcedure
\end{algorithmic}
\end{algorithm}
See Figure \ref{fig:fig1}. Here is how you add footnotes. \footnote{Sample of the first footnote.}
\lipsum[11] 
\begin{figure}
\centering\includegraphics[width=0.8\textwidth]{Fig/Comparison-between-Ecoli-bacterium-and-Raspberry-Pi-controlled-IoT-device-Components-of.png} 
  \caption{Sample figure caption.}
\label{fig:fourchamber}\end{figure} 
\begin{figure}
\centering\includegraphics[width=0.8\textwidth]{Fig/6-Figure3-1.png} 
  \caption{Sample figure caption.}
\label{fig:fourchamber}\end{figure} 
\begin{figure}
\centering\includegraphics[width=0.8\textwidth]{Fig/DNABOT.png} 
  \caption{Sample figure caption.}
\label{fig:fourchamber}\end{figure}
\begin{figure}
\centering\includegraphics[width=0.8\textwidth]{Fig/Conaimi.png} 
  \caption{Sample figure caption.}
\label{fig:fourchamber}\end{figure} 
\begin{figure}
\centering\includegraphics[width=0.8\textwidth]{Fig/DNAProcess.jpg} 
  \caption{Sample figure caption.}
\label{fig:fourchamber}\end{figure} 
\begin{figure}
  \centering
  \fbox{\rule[-.5cm]{4cm}{4cm} \rule[-.5cm]{4cm}{0cm}}
  \caption{Sample figure caption.}
  \label{fig:fig1}
\end{figure}

\subsection{Tables}
\lipsum[12]
See awesome Table~\ref{tab:table}.

\begin{table}
 \caption{Sample table title}
  \centering
  \begin{tabular}{lll}
    \toprule
    \multicolumn{2}{c}{Part}                   \\
    \cmidrule(r){1-2}
    Name     & Description     & Size ($\mu$m) \\
    \midrule
    Dendrite & Input terminal  & $\sim$100     \\
    Axon     & Output terminal & $\sim$10      \\
    Soma     & Cell body       & up to $10^6$  \\
    \bottomrule
  \end{tabular}
  \label{tab:table}
\end{table}

\subsection{Lists}
\begin{itemize}
\item Lorem ipsum dolor sit amet
\item consectetur adipiscing elit. 
\item Aliquam dignissim blandit est, in dictum tortor gravida eget. In ac rutrum magna.
\end{itemize}
\section{Sci fiction vision}
\label{sec:headings}
las biocryptocurrenci trabajaran dentro de la celula, estas le daran valor a las celulas, por los mecanismos incentiviadores que posseen.
trabajaran con los mecanimos celulares de la celula, algun mecanismo tendra la proof  para discentivar los que se este creando  . La descentralizacion que desarrolla la celula provee un nuevo tipo--- stock creando otro tipo de ecosistema.el ecosystema tiene un desarrollo en las blockchain sales diviendolo en utilituy tokens y security tokens. En este sebntido las biocurrencies tendran el sentimiento de estar reinventando la rueda.Pero talvez los investigadores podran implementar protocolos muy eficientes y strategias basadas en otro tipo de descentralizacion.Evitando el monopolio de nodos que corren en la redes descentralizadas, evitando la acumulacion the stake.Este nuevo tipo de valor podria darle valor a diferrentes cosas, no a la idea a tokenizar cosas. si no que las cosas por sus propiedades naturales creen su propio valor dentro de los mercados.(La idea alquimista de poder convertir el estaño en oro).Las cryptocurrencies dan la opcion crear valor en cosas que esta respaldas de nada y otras opciones de tokenizar utilidades o algun tipo valor con un mercado.Las biocurrencies o biocryptocurrencies abriran otro tipo de mercado dado que la idea de tokenizar se extendera a un valor natural ontologico.
The biocryptocurrenci will work inside the cell, they will give value to the cells, by the incentive mechanisms they possess.
They will work with the cell mechanics of the cell, some mechanism will have the test to discourage those that are being created. The decentralization that develops the cell provides a new type --- stock creating another type of ecosystem. The ecosystem has a development in blockchain sales by dividing it into utility tokens and security tokens. In this sense, biocurrences will have the feeling of reinventing the wheel, but perhaps researchers will be able to implement very efficient protocols and strategies based on another type of decentralization. This new type of value can give value to different things, not the idea of ​​tokenizing things. if not things by their natural properties create their own value within the markets. (The alchemist idea of ​​being able to convert tin into gold). Cryptocurrencies give the option of creating value in things that are backed by nothing and other options of tokenizing profits or some type of value with a market. Bio currencies or cryptocurrencies will open another type of market since the idea of ​​tokenizing will extend a natural ontological value


\bibliographystyle{plain}
\bibliography{references.bib}




\end{document}
